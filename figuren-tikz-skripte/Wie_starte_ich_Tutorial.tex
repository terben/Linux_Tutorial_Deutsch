\documentclass{standalone}
\usepackage{tikz}
\usetikzlibrary{positioning,shapes.symbols}
%
\begin{document}

\begin{tikzpicture}
  \node[anchor=south west,inner sep=0] (image) at (0,0) {\includegraphics[width=5cm]{figuren/start_tutorial.png}};
  %
  \begin{scope}[x={(image.south east)},y={(image.north west)}]
%    \draw[help lines, very thin, step=0.02] (0,0) grid (1,1);
%    \draw[help lines,thin,xstep=.1,ystep=.1] (0,0) grid (1,1);
%    \foreach \x in {0,1,...,9} { \node [anchor=north] at (\x/10,0) {0.\x}; }
%    \foreach \y in {0,1,...,9} { \node [anchor=east] at (0,\y/10) {0.\y}; }

     \node[fill = blue!10, font=\tiny, align = center] (tutorial) at (0.45, 0.65) {Tutorial \\ Anleitung};
%    \node[fill = red!10, font=\tiny, align = center] (terminal) at (0.6, 1.0) {Linux-Terminal \\ starten};
%    \draw[red, thick] (0.8, 0.15) rectangle (0.88, 0.23);
    \draw[blue, thick] (0.03, 0.26) rectangle (0.36, 0.34);
    \draw[-latex, blue, thick] (tutorial) to (0.19, 0.35);
%    \draw[-latex, red, thick] (terminal) to (0.84, 0.23);
  \end{scope}
\end{tikzpicture}

\end{document}, anchor = west