% Author: Valeria Borodin
\documentclass[border={5pt 40pt 180pt 130pt}, % left bottom right top
  svgnames]{standalone}
%\documentclass{standalone}
%%%<
\usepackage{verbatim}
%%%>
\usepackage{tikz}
\usetikzlibrary{positioning}
\usepackage{listings}
\definecolor{whitesmoke}{rgb}{0.96, 0.96, 0.96}
\tikzstyle{every picture}+=[remember picture]

\lstset{%
  frame            = tb,    % draw frame at top and bottom of code block
  tabsize          = 1,     % tab space width
  numbers          = left,  % display line numbers on the left
  framesep         = 3pt,   % expand outward
  framerule        = 0.4pt, % expand outward
  commentstyle     = \color{blue},      % comment color
  keywordstyle     = \color{green},       % keyword color
  stringstyle      = \color{darkred},    % string color
  backgroundcolor  = \color{white}, % backgroundcolor color
  showstringspaces = false,              % do not mark spaces in
                                         % strings
  otherkeywords    = {in},
}
\begin{document}
%
\begin{lstlisting}[language = bash, numbers = none, escapechar = !,
    basicstyle = \ttfamily\bfseries, linewidth = .6\linewidth]
 for!
   \tikz[baseline] \node [fill=blue!10, anchor=base] (a) {DATEI};
 !in !
   \tikz[baseline] \node [fill=red!10, anchor=base] (b) {Jupiter.txt Saturn.txt};!
 do
   !
   \tikz[baseline] \node[fill=orange!10, anchor=base, align=left] (c) {sort -g -k4 \$\{DATEI\} | head -n 1};!
 done
\end{lstlisting}
\begin{tikzpicture}[remember picture, overlay,
    every edge/.append style = { ->, thick, >=stealth,
                                  DimGray, dashed, line width = 1pt },
    every node/.append style = { align = center, minimum height = 10pt},
                  text width = 2.5cm ]
  \node [above = .5cm of a,text width = 2.2cm, fill=blue!10, font=\bfseries]
                             (A) {Schleifen-\\variable};
  \node [above = .5cm of b,text width = 2.2cm, fill=red!10, font=\bfseries]
                             (B) {Schleifen-\\liste};
  \node [below = 1.0cm of c,text width = 3cm, fill=orange!10, font=\bfseries]
                             (C) {Schleifenbefehle};
  %
  \node[above = 4.9cm of c, text width = 6cm] {Struktur der \texttt{for}-Schleife};
\end{tikzpicture}
\end{document}
